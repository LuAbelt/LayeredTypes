\section{Introduction}
\label{sec:introduction}
\subsection{Context}
\label{ssec:context}
Type systems are a core part of most modern programming languages. Most programmers will be familiar with the concept of simple static type systems as they are present in languages such as C++, Java, Rust or Haskell, that assign types to variables and functions and checks that types are used in a correct way. Using static type systems bears a lot of benefits as it can already prevent a multitude of runtime errors.

However, static type systems alone cannot prevent runtime errors alone as they do not present strong enough guarantees. Therefore many other type systems have been proposed over the years. One example of this are \textit{Refinement Types}, also referred to as \textit{Liquid Types}. These allow us to annotate the traditional type system with additional logical predicates in order to ensure semantic properties, which can be checked statically at compile time. Using these techniques also allows us to define pre- and post-conditions on functions' arguments and return values. By properly defining these contracts a developer can clearly communicate and moreover enforce how other developers' code is allowed to interface with their implementations. Examples for constraints might include any of the following:

\begin{itemize}
	\item Ensuring values are in certain bounds
	\item Ensuring that a list is non-empty
	\item Ensuring that data structures certain to specific semantic properties e.g. that they are sorted
\end{itemize}

All in all, refinement types are valuable building blocks to creating reliable software systems. Using them can minimize the risk of any unhandled run-time errors occuring which might lead to undesired or even undefined behavior.

Refinement Types are but one example of the many different typesystems that have been introduced over the years. In this paper we want to discuss how it is possible to combine different type systems and combine their advantages.

\subsection{Motivation}
\label{ssec:motivation}

While various type systems have been proposed over the years, using even one of them in an existing program is not a trivial task. In order to use these more advanced type systems, programmers often depend on specialized tools such as separate type checkers or compilers. Depending on the specific implementation these tools may or may not be easily integratable into existing development workflows. It gets even more complicated when trying to combine multiple type systems. In order to use multiple type systems, programmers then rely on a multitude of different tools that often cannot be arbitrarily combined. This can lead to a lot of overhead and complexity in the development process.

Our motivation for this paper was to propose an approach, that allows to easily extend and combine existing type systems, with the goal of making them more accessible to programmers. We want to achieve this by providing a framework that allows to easily combine different type systems and ensures a consistent type checking process.

The original motivation for this paper however, stems from a problem often prevalent while using refinement types; While Refinement Types vastly imrpove traditional type systems, current implementations still expose some drawbacks in their practical usage. One of these challenges stems from the fact that during the static analysis, each refinement is treated as a single monolithic requirement that needs to be fulfilled. However in reality, a refinement might combine different properties of a type that may benefit from being treated individually.

One example for such a refinement in LiquidHaskell is shown in \Cref{lst:combined_lt}. There, we define a function that extracts the maximum value of a (descendingly) ordered, non-empty list. In this example, this operation is equivalent to simply extracting the \texttt{head} of a list. While we only define one refinement, it actually operates on two different properties of a list. First, it imposes a restriction on it's \texttt{size} by requiring it to be non-empty, while the second restriction imposes an order on the list elements.

\lstinputlisting[firstline=6,caption={Example for a function with a combined Refinement Type in LiquidHaskell},label={lst:combined_lt}]{CombinedLT.hs}

While the refinement concerns two different aspects of the list that are independent from each other, current static analysis implementations of Liquid Types will treat it only as a single combined requirement to be fulfilled. However especially developers could benefit from a system that treats them independently: When writing code that interfaces with already existing codes such a separated approach may give a developer more fine-grained information on which requirements they still need to fulfill in their implementation.

Effectively one could treat such different requirements as separate \textit{Layers} that can be verified independently. Additionally with a proper implementation using such layers may also result in an implementation benefit, that different Layers can be defined individually and combined arbitrarily, as opposed to the example in Listing \Cref{lst:combined_lt} where the combination of the two traits had to be explicitly defined as an refinement on it's own.

One other possible usage of such an iterative approach is that it could allow to more flexibly add different static analysis approaches on top of each other. For example \cite{liquidate-assets} have shown how to use liquid types in Haskell to statically perform cost analysis of different resources such as memory consumption or computational complexity. In their approach they showed that liquid types can be used to reason about their upper and lower bounds. However, in their work they only analysed one resource at a time. However, using a layered approach would allow to easily combine these different resource analyses to eg. track and reason about both memory usage and computational complexity.

\subsection{Problem Definition}
\label{ssec:problem_definition}

Current verification approaches usually focus on verifying a single property of a program at a time. However, in practice, one might needs to verify different properties of the program for different purposes. In current implementations, it is not trivially possible to combine these different verification passes. One specific examplefor this can be found when looking at current implemetnations of Liquid Types. In these, one can only define one liquid type for variables, arguments or return types. This forces us to combine all properties we want to statically check and verify into one joint predicate. Using such an approach has the drawback that all properties are checked simultaneously which might make it difficult for a developer to identify due to which property a type error occurs. Additionally previously declared liquid types may not be reused when wanting to refine them further.

With this paper, we therefore propose a new incremental verification approach that we will refer to as \textit{Layered Types}. The goal of this approach is it for a program to be possible to define individual layers for the verification process that are verified individually. By defining dependencies between those layers, it is possible to define a verification order that ensures that all layers are verified in the correct order. This allows us to define different properties of a program that can be verified independently from each other.

With this paper, we want to tackle the following research questions:

\begin{enumerate}
	\item RQ1: Can we define a layered type system that allows to treat different properties of a program independently from each other?
	\item RQ2: Can we use such a layered type system to improve challenges that are present in current Liquid Type implementations?
\end{enumerate}

While in this paper, we will focus on specific Use-Cases that aim towards using it with Liquid Types, we believe that the proposed approach can be applied to other verification techniques as well. For example, we believe that the proposed approach can be extended to verify all sorts of different properties and typesystems such as e.g. \textit{Linear Types} \cite{linear_types} or \textit{Dependent Types} \cite{dependent_types}.

We also want to make an important distinction on the terminology used within the scope of this paper: While on first glance this approach could be described as an approach to \textit{Gradual Verification}, we will explicitly refer to our approach as \textit{Layered} to distinguish it from \textit{Gradual Verification} techniques. While the latter deal with the gradient between incomplete and complete information (in that case on types), our approach is rather concerned not with what information is available on the typing but rather how it is processed. However, we believe that our approach can also be used to implement gradual verification techniques.

\subsection{Impact}
\label{ssec:impact}

The main impact of this paper is to showcase an approach of how different, dependent or independent, type systems can be combined using a single framework without the need to rely on multiple different tools. This allows to combine the advantages of different type systems and allows to use them in a more flexible way. Additionally, we believe that this approach can be used to improve the usability of Liquid Types and other verification techniques. Specifically, we believe our approach has the potential to have the following impacts on Liquid Types in practice:

\paragraph{Improved Error Messages}

One current challenge when using liquid types is that all predicates for a type are checked simultaneously. This can make it difficult to identify which predicate caused a type liquid error. By separating the predicates into separate \textit{Layers} and verifying them individually, it is possible to precisely identify the predicate causing the type error. We expect this to help developers to more easily identify and fix these type errors in their code.

\paragraph{Improved Usability and Reusability}

In current Liquid Type implementations, types that require multiple predicates are defined as a single type. This may make it difficult for a developer to directly identify which properties a specific liquid type ensures. By separating different properties into a more fine-grained layered approach, this may make it easier to identify these properties. Additionally, by separating the predicates into different layers, it is possible to reuse liquid types in a more flexible way. For example, if a developer wants to refine a previously defined liquid type, they can now do so by adding a new layer to it, instead of having to redefine the entire type.

\paragraph{Improved Composability}

In addition to the improved reusability of our approach, we also expect it to improve the composability of different liquid types. With our suggested approach, new refinements can simply be created by composing already existing layers. This vastly increases modularity of the liquid type approach. It also may allow to re-use layers that have already been defined by other developers for ones own implementation.

\paragraph{Improved Modularity}

By separating the predicates into different layers, it is possible to reuse previously defined liquid types in a more flexible way. For example, if a developer wants to refine a previously defined liquid type, they can now do so by adding a new layer to it, instead of having to redefine the entire type. Additionally, this also supports the seperation of concerns when collaborating with other developers. In current implementations, when interfacing with other developers' code, one needs to be aware of all the refinements required by such code in order to interact with it. When different properties are divided into proper layers, this is no longer required.

\subsection{Approach}
\label{ssec:approach}

We will start by giving a brief overview of the general principles of refinement types and liquid types in general in section \ref{sec:background}. Additionally, we will discuss the current state of the art in liquid types and gradual verification in section \ref{sec:related}. 

In section \ref{sec:implementation} we will first give a more detailed description of our approach and define a simple functional language that we will use for further demonstration. We will then describe in detail the proposed architecture of our layered type checker and our layered verification framework.

In section \ref{sec:use_cases} we will present two specific use cases. One of which will showcase a way of combining different independent type systems using our layered approach. The other use case will showcase how our approach can be used to verify different properties independently in a liquid type system.

In section \ref{sec:evaluation} we will discuss the limitations of our approach and possible extensions. Additionally, we will evaluate how our approach compares to using a pure, non-layered, verification approach.

\subsection{Contributions}
\label{ssec:contributions}

With this paper we deliver the following contributions:

\begin{enumerate}
	\item We propose a new approach to incremental verification that we will refer to as \textit{Layered Types}.
	\item We present an architecture for a layered verification framework that allows to verify different properties of a program independently from each other.
	\item We present two specific use cases that showcase how our approach can be used to verify different properties independently in a liquid type system.
	\item We deliver a prototype implementation of our layered type checker framework, including some example implementations for layers in said framework. Our implementation is available at \url{https://www.github.com/LuAbelt/LayeredTypes}.
\end{enumerate}
